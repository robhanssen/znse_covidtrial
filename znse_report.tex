% Options for packages loaded elsewhere
\PassOptionsToPackage{unicode}{hyperref}
\PassOptionsToPackage{hyphens}{url}
%
\documentclass[
]{article}
\usepackage{amsmath,amssymb}
\usepackage{lmodern}
\usepackage{ifxetex,ifluatex}
\ifnum 0\ifxetex 1\fi\ifluatex 1\fi=0 % if pdftex
  \usepackage[T1]{fontenc}
  \usepackage[utf8]{inputenc}
  \usepackage{textcomp} % provide euro and other symbols
\else % if luatex or xetex
  \usepackage{unicode-math}
  \defaultfontfeatures{Scale=MatchLowercase}
  \defaultfontfeatures[\rmfamily]{Ligatures=TeX,Scale=1}
\fi
% Use upquote if available, for straight quotes in verbatim environments
\IfFileExists{upquote.sty}{\usepackage{upquote}}{}
\IfFileExists{microtype.sty}{% use microtype if available
  \usepackage[]{microtype}
  \UseMicrotypeSet[protrusion]{basicmath} % disable protrusion for tt fonts
}{}
\makeatletter
\@ifundefined{KOMAClassName}{% if non-KOMA class
  \IfFileExists{parskip.sty}{%
    \usepackage{parskip}
  }{% else
    \setlength{\parindent}{0pt}
    \setlength{\parskip}{6pt plus 2pt minus 1pt}}
}{% if KOMA class
  \KOMAoptions{parskip=half}}
\makeatother
\usepackage{xcolor}
\IfFileExists{xurl.sty}{\usepackage{xurl}}{} % add URL line breaks if available
\IfFileExists{bookmark.sty}{\usepackage{bookmark}}{\usepackage{hyperref}}
\hypersetup{
  pdftitle={Are Zn and Se-based supplement a relief for COVID-19 vaccination side-effect?},
  pdfauthor={Rob Hanssen},
  hidelinks,
  pdfcreator={LaTeX via pandoc}}
\urlstyle{same} % disable monospaced font for URLs
\usepackage[margin=1in]{geometry}
\usepackage{color}
\usepackage{fancyvrb}
\newcommand{\VerbBar}{|}
\newcommand{\VERB}{\Verb[commandchars=\\\{\}]}
\DefineVerbatimEnvironment{Highlighting}{Verbatim}{commandchars=\\\{\}}
% Add ',fontsize=\small' for more characters per line
\usepackage{framed}
\definecolor{shadecolor}{RGB}{248,248,248}
\newenvironment{Shaded}{\begin{snugshade}}{\end{snugshade}}
\newcommand{\AlertTok}[1]{\textcolor[rgb]{0.94,0.16,0.16}{#1}}
\newcommand{\AnnotationTok}[1]{\textcolor[rgb]{0.56,0.35,0.01}{\textbf{\textit{#1}}}}
\newcommand{\AttributeTok}[1]{\textcolor[rgb]{0.77,0.63,0.00}{#1}}
\newcommand{\BaseNTok}[1]{\textcolor[rgb]{0.00,0.00,0.81}{#1}}
\newcommand{\BuiltInTok}[1]{#1}
\newcommand{\CharTok}[1]{\textcolor[rgb]{0.31,0.60,0.02}{#1}}
\newcommand{\CommentTok}[1]{\textcolor[rgb]{0.56,0.35,0.01}{\textit{#1}}}
\newcommand{\CommentVarTok}[1]{\textcolor[rgb]{0.56,0.35,0.01}{\textbf{\textit{#1}}}}
\newcommand{\ConstantTok}[1]{\textcolor[rgb]{0.00,0.00,0.00}{#1}}
\newcommand{\ControlFlowTok}[1]{\textcolor[rgb]{0.13,0.29,0.53}{\textbf{#1}}}
\newcommand{\DataTypeTok}[1]{\textcolor[rgb]{0.13,0.29,0.53}{#1}}
\newcommand{\DecValTok}[1]{\textcolor[rgb]{0.00,0.00,0.81}{#1}}
\newcommand{\DocumentationTok}[1]{\textcolor[rgb]{0.56,0.35,0.01}{\textbf{\textit{#1}}}}
\newcommand{\ErrorTok}[1]{\textcolor[rgb]{0.64,0.00,0.00}{\textbf{#1}}}
\newcommand{\ExtensionTok}[1]{#1}
\newcommand{\FloatTok}[1]{\textcolor[rgb]{0.00,0.00,0.81}{#1}}
\newcommand{\FunctionTok}[1]{\textcolor[rgb]{0.00,0.00,0.00}{#1}}
\newcommand{\ImportTok}[1]{#1}
\newcommand{\InformationTok}[1]{\textcolor[rgb]{0.56,0.35,0.01}{\textbf{\textit{#1}}}}
\newcommand{\KeywordTok}[1]{\textcolor[rgb]{0.13,0.29,0.53}{\textbf{#1}}}
\newcommand{\NormalTok}[1]{#1}
\newcommand{\OperatorTok}[1]{\textcolor[rgb]{0.81,0.36,0.00}{\textbf{#1}}}
\newcommand{\OtherTok}[1]{\textcolor[rgb]{0.56,0.35,0.01}{#1}}
\newcommand{\PreprocessorTok}[1]{\textcolor[rgb]{0.56,0.35,0.01}{\textit{#1}}}
\newcommand{\RegionMarkerTok}[1]{#1}
\newcommand{\SpecialCharTok}[1]{\textcolor[rgb]{0.00,0.00,0.00}{#1}}
\newcommand{\SpecialStringTok}[1]{\textcolor[rgb]{0.31,0.60,0.02}{#1}}
\newcommand{\StringTok}[1]{\textcolor[rgb]{0.31,0.60,0.02}{#1}}
\newcommand{\VariableTok}[1]{\textcolor[rgb]{0.00,0.00,0.00}{#1}}
\newcommand{\VerbatimStringTok}[1]{\textcolor[rgb]{0.31,0.60,0.02}{#1}}
\newcommand{\WarningTok}[1]{\textcolor[rgb]{0.56,0.35,0.01}{\textbf{\textit{#1}}}}
\usepackage{graphicx}
\makeatletter
\def\maxwidth{\ifdim\Gin@nat@width>\linewidth\linewidth\else\Gin@nat@width\fi}
\def\maxheight{\ifdim\Gin@nat@height>\textheight\textheight\else\Gin@nat@height\fi}
\makeatother
% Scale images if necessary, so that they will not overflow the page
% margins by default, and it is still possible to overwrite the defaults
% using explicit options in \includegraphics[width, height, ...]{}
\setkeys{Gin}{width=\maxwidth,height=\maxheight,keepaspectratio}
% Set default figure placement to htbp
\makeatletter
\def\fps@figure{htbp}
\makeatother
\setlength{\emergencystretch}{3em} % prevent overfull lines
\providecommand{\tightlist}{%
  \setlength{\itemsep}{0pt}\setlength{\parskip}{0pt}}
\setcounter{secnumdepth}{-\maxdimen} % remove section numbering
\usepackage{amsmath}
\usepackage{booktabs}
\usepackage{caption}
\usepackage{longtable}
\ifluatex
  \usepackage{selnolig}  % disable illegal ligatures
\fi

\title{Are Zn and Se-based supplement a relief for COVID-19 vaccination
side-effect?}
\author{Rob Hanssen}
\date{1/30/2022}

\begin{document}
\maketitle

\hypertarget{background}{%
\section{Background}\label{background}}

Based on some medical literature and a comment from a co-worker, I've
been looking at the effect of zinc and selenium on the occurrence of
symptoms after COVID-19 vaccination (fever, chills, exhaustions, etc.).

\hypertarget{the-clinical-trial}{%
\section{The ``clinical trial''}\label{the-clinical-trial}}

A treatment consisting of additional Zn/Se was administered three days
before the vaccination and two days after and the occurrence/absence of
symptoms was recorded. The control group (no treatment) was based on
people volunteering their experiences without treatment, while the
treated group used treatment from a single bottle of multi-vitamin
pills. This means it's certainly not double-blind (the author was part
of the treated group).

\hypertarget{first-results}{%
\section{First results}\label{first-results}}

Results are in the table in the appendix. All people with treatments had
no symptoms. Some people without treatment had no symptoms on the first
vaccination, but were affected in subsequent shots.

\begin{Shaded}
\begin{Highlighting}[]
\NormalTok{trial }\SpecialCharTok{\%\textgreater{}\%}
\NormalTok{  ggplot }\SpecialCharTok{+}
  \FunctionTok{aes}\NormalTok{(}\AttributeTok{x =}\NormalTok{ zn\_se, }\AttributeTok{y =}\NormalTok{ symptoms, }\AttributeTok{color =}\NormalTok{ vax\_number) }\SpecialCharTok{+} 
  \FunctionTok{geom\_jitter}\NormalTok{(}\AttributeTok{width =}\NormalTok{ .}\DecValTok{2}\NormalTok{, }\AttributeTok{height =}\NormalTok{ .}\DecValTok{2}\NormalTok{) }\SpecialCharTok{+} 
  \FunctionTok{labs}\NormalTok{(}\AttributeTok{x =} \StringTok{"Zn/Se treatment"}\NormalTok{,}
       \AttributeTok{y =} \StringTok{"Symptoms"}\NormalTok{,}
       \AttributeTok{color =} \StringTok{"Vaccine}\SpecialCharTok{\textbackslash{}n}\StringTok{shot order"}\NormalTok{)}
\end{Highlighting}
\end{Shaded}

\includegraphics{znse_report_files/figure-latex/unnamed-chunk-2-1.pdf}

\hypertarget{discussion}{%
\section{Discussion}\label{discussion}}

Given the small dataset (25 observations over 10 patients), it was
considered prudent to only look at the effect of the treatment and
forego analysis of the influence of the sequence of injection.

\begin{Shaded}
\begin{Highlighting}[]
\NormalTok{Xsqr }\OtherTok{\textless{}{-}}
\NormalTok{  janitor}\SpecialCharTok{::}\FunctionTok{tabyl}\NormalTok{(trial, zn\_se, symptoms) }\SpecialCharTok{\%\textgreater{}\%}
\NormalTok{  janitor}\SpecialCharTok{::}\FunctionTok{chisq.test}\NormalTok{() }\SpecialCharTok{\%\textgreater{}\%}
\NormalTok{  broom}\SpecialCharTok{::}\FunctionTok{tidy}\NormalTok{()}

\NormalTok{N }\OtherTok{\textless{}{-}} \FunctionTok{nrow}\NormalTok{(trial)}

\NormalTok{effectsize }\OtherTok{\textless{}{-}} \FunctionTok{sqrt}\NormalTok{(Xsqr}\SpecialCharTok{$}\NormalTok{statistic }\SpecialCharTok{/}\NormalTok{ N)}
\NormalTok{power }\OtherTok{\textless{}{-}}\NormalTok{ pwr}\SpecialCharTok{::}\FunctionTok{pwr.chisq.test}\NormalTok{(}\AttributeTok{w =}\NormalTok{ effectsize,}
                             \AttributeTok{N =}\NormalTok{ N, }
                             \AttributeTok{df =}\NormalTok{ Xsqr}\SpecialCharTok{$}\NormalTok{parameter) }\SpecialCharTok{\%\textgreater{}\%} 
\NormalTok{  broom}\SpecialCharTok{::}\FunctionTok{tidy}\NormalTok{()}
\end{Highlighting}
\end{Shaded}

A \(\chi^2\) analysis shows that \(\chi^2\) = 10.15, yielding of a p
value of 0.001, which is an indication that considerable reason to
disregard the \emph{null} hypothesis of equality of treatment \emph{vs}
non-treatment is supported. Further analysis of the power of the test
(0.890 at significance level 0.050) shows that applying this treatment
has a high likelihood of preventing symptons after a COVID-19
vaccination.

Further trials at a much larger scale would be beneficial of proving the
efficacy of this treatment, since a trial at this level can only be
considered anectdotal at best.

\hypertarget{conclusions}{%
\section{Conclusions}\label{conclusions}}

A combination of Zn and Se supplements administered through common
supplements in the days before a COVID-19 vaccination could be
beneficial in relieving some of the side-effects (\emph{e.g.} fever,
chills or exhaustion).

\hypertarget{disclaimer}{%
\section{Disclaimer}\label{disclaimer}}

I'm not a medical professional and this ``clinical trial'' does not
adhere to any standards. I don't want to start a discussion on COVID-19,
need for vaccines, microchips, or anything not related to my data. I'm
not paid by \emph{Big Selenium}.

\hypertarget{appendix}{%
\section{Appendix}\label{appendix}}

\begin{verbatim}
## Joining, by = "patient"
\end{verbatim}

\captionsetup[table]{labelformat=empty,skip=1pt}
\begin{longtable}{lrlcccc}
\toprule
patientcode & age & gender & vax\_number & vax\_type & zn\_se & symptoms \\ 
\midrule
Patient 1 & 12 & M & 1 & Pfizer & Treatment & No symptoms \\ 
Patient 1 & 12 & M & 2 & Pfizer & Treatment & No symptoms \\ 
Patient 2 & 40 & M & 1 & Moderna & No treatment & No symptoms \\ 
Patient 2 & 40 & M & 2 & Moderna & No treatment & Symptoms \\ 
Patient 2 & 40 & M & 3 & Moderna & No treatment & Symptoms \\ 
Patient 3 & 49 & F & 1 & Pfizer & No treatment & Symptoms \\ 
Patient 3 & 49 & F & 2 & Pfizer & No treatment & Symptoms \\ 
Patient 4 & 72 & M & 1 & Pfizer & No treatment & No symptoms \\ 
Patient 4 & 72 & M & 2 & Pfizer & No treatment & Symptoms \\ 
Patient 4 & 72 & M & 3 & Pfizer & No treatment & Symptoms \\ 
Patient 5 & 45 & F & 1 & Moderna & Treatment & No symptoms \\ 
Patient 5 & 45 & F & 2 & Moderna & Treatment & No symptoms \\ 
Patient 5 & 47 & M & 3 & Pfizer & Treatment & No symptoms \\ 
Patient 6 & 40 & F & 1 & Moderna & No treatment & No symptoms \\ 
Patient 6 & 40 & F & 2 & Moderna & No treatment & Symptoms \\ 
Patient 7 & 10 & F & 1 & Pfizer & No treatment & Symptoms \\ 
Patient 8 & 47 & M & 1 & Pfizer & Treatment & No symptoms \\ 
Patient 8 & 47 & M & 2 & Pfizer & Treatment & No symptoms \\ 
Patient 8 & 47 & M & 3 & Pfizer & Treatment & No symptoms \\ 
Patient 9 & 50 & M & 1 & Pfizer & No treatment & Symptoms \\ 
Patient 9 & 50 & M & 2 & Pfizer & No treatment & Symptoms \\ 
Patient 9 & 50 & M & 3 & Pfizer & No treatment & No symptoms \\ 
Patient 10 & 67 & F & 1 & Pfizer & No treatment & Symptoms \\ 
Patient 10 & 67 & F & 2 & Pfizer & No treatment & Symptoms \\ 
Patient 10 & 67 & F & 3 & Pfizer & Treatment & No symptoms \\ 
 \bottomrule
\end{longtable}

\end{document}
